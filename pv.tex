\documentclass[a4paper, 12pt]{report}
\usepackage[utf8]{inputenc} % UTF-8 encoding
\usepackage[french]{babel} % Language support
\usepackage[T1]{fontenc} % Times New Roman
% \usepackage[latin1]{inputenc} % ISO 8859-1
\usepackage{lmodern} % modern fonts
\usepackage{xcolor} % color
\usepackage{geometry} % Page size
\usepackage{enumitem}
\usepackage{titlesec, blindtext, color} % blindtext for titles
\usepackage{graphicx} % Images
\usepackage{hyperref}

\begin{document}

\begin{titlepage}
    \begin{center}
        \vspace*{\fill}


        \textcolor{gray}{\textbf{\huge{PV de livraison}}}

        \vspace{0.5cm}

        \textsl{\large{Projet <<Gestion de Stage>>}}

        \vspace{1.5cm}

        \rule{\linewidth}{0.15mm}
        \textbf{Auteurs}

        \vspace{0.5cm}

        \small{\textit{Aymerick LAURETTA-PERONNE}} \\
        \small{\textit{Benjamin NIDDAM}} \\
        \small{\textit{Théo FIGINI}} \\
        \small{\textit{Nicolas GOUWY}} \\
        \small{\textit{Syoan ODOUHA}} \\
        \small{\textit{Isaïe CLARIN}}
        \rule{\linewidth}{0.15mm}

        \vspace{2.5cm}

        \textbf{Version <<1.0>>}

        \vfill
        \textbf{\today {}}
    \end{center}
    \vspace*{\fill}
\end{titlepage}

\section*{Introduction}

L'objectif de ce projet est de réaliser une application de mise en relation d'entreprises et d'étudiants dans l'optique de favoriser
et faciliter la recherche de stage, d'emploi ou d'alternance.

\newpage
\section*{Description de la livraison}
\subsection*{Documents}

\begin{itemize}
    \item Document de spécifications
    \item Document de conception
    \item Bilan du projet
\end{itemize}


Le code source est disponible \href{https://github.com/BBIITS/Gestion_de_stages}{ici}. (GitHub)


\subsection*{Fonctionnalités livrés}

Actuellement l'application permet la création de deux types de compte : étudiant et entreprise. Les utlisateurs ayant déjà un compte
peuvent se connecter. Les entreprises peuvent ajouter des offres de stage et les étudiants peuvent postuler à ces offres. Les entreprises
peuvent réagir à ces candidatures: accepter ou refuser. Les entreprises ainsi que les etudiants peuvent supprimer respectivement leurs
offres et leurs candidatures. Et enfin, les utilisateurs ont la possibilité de se déconnecter.


\subsection*{Informations relatives à la mise en place de l'application}
Pour pouvoir héberger et déployer l'application sur le serveur, il faut avoir nodejs et npm installés en leur version les plus récentes.
Avec ceci, il faudra installer la libraire $"PM2"$ en global sur votre serveur ainsi que le SGBDR $"mariadb"$ pour la base de données. Une fois ces
outils installés, il faut via le SGBDR créer une base de données nomée $"gestion\_de\_stages"$.
Une fois fait, il ne reste plus qu'à acceder au dossier de l'application,
%le .env

et lancer la commande $"npm install"$ pour installer les dépendances
puis $"npm \: run \: start:prod"$ pour démarrer l'application. Cette commande va démarrer l'application en mode production et créer toutes les tables
nécessaires à l'application dans la base de données.


\section*{Commentaires}

Il ne manque que deux fonctionnalités à ajouter dans l'application pour que cette derniere respecte complètement
le cahier des charges:
\begin{itemize}
    \item la gestion des notifications
    \item permettre aux utilisateurs d'acceder à la liste des offres de stage sans être inscrits
\end{itemize}

\end{document}